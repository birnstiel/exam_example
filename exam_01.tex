% set option 'answers' do show solution
\documentclass[addpoints,12pt]{exam}
\usepackage[english]{babel}
\usepackage[margin=1in,inner=0.6in]{geometry}
\usepackage[utf8]{inputenc}
\usepackage[T1]{fontenc}
\usepackage[dvipsnames]{xcolor}
\usepackage{mathtools}
\usepackage{amssymb}
\usepackage{multicol}
\usepackage{siunitx}
\usepackage[capitalise]{cleveref}
\usepackage{wrapfig}
\usepackage{caption} % for putting captions in minipages
\usepackage{graphicx}

% define some variables
\newcommand{\examtitle}{Example Exam WS 2024/25}
\newcommand{\examdate}{Dec. 09, 2024}
\newcommand{\examauthor}{Prof. Dr. T. Birnstiel}
\newcommand{\examheader}{Example Exam}

% do not indent paragraphs
\parindent0pt

%%%%% to make boxes

\usepackage{mdframed}
\newmdenv [linecolor=black,backgroundcolor=gray!15,
    frametitle={Note},leftmargin=1cm,
    rightmargin=1cm]{infobox}

%%%%% define the header and footer

\makeatletter
\@ifclasswith{exam}{answers}{\newcommand{\mytitle}[1]{\textcolor{red}{Solution: #1}}}{\newcommand{\mytitle}[1]{#1}}
\makeatother

\makeatletter
% command that makes a new page also in the solution-version
\@ifclasswith{exam}{answers}{\newcommand{\examnewpage}{\clearpage}}{\newcommand{\examnewpage}{\clearpage}}
% new page only in exam, not in the solution
\@ifclasswith{exam}{answers}{\newcommand{\examextrapage}{\clearpage}}{\newcommand{\examextrapage}{\clearpage \makeemptybox{\stretch{1}} \clearpage}}
\makeatother

\pagestyle{headandfoot}
\runningheadrule
\firstpageheader
{LMU München}
{
    \begin{tabular}[t]{@{}c@{}} \mytitle{\textsc{\examtitle}}\\\begin{small}\examdate\end{small}\end{tabular}
}
{\examauthor}
\runningheader{\examheader}
{}
{Page \thepage\ / \numpages}
\firstpagefooter{}{}{}
\runningfooter{}{}{}

%%%%%%%

% set the height between answer-lines
\setlength{\linefillheight}{2em}

% set the grid properties
\colorgrids % turns on color of the grid
\definecolor{GridColor}{gray}{.8}
\setlength{\gridsize}{5mm}
\setlength{\gridlinewidth}{0.1pt}

% to shade solution boxes:
\shadedsolutions

% to have the points on the right margin
\pointsinrightmargin

% make boxes/brackets around the points instead of parantheses
\boxedpoints
%\bracketedpoints

%%%%%% switch to sans-serif as default font and use helvetica
\renewcommand{\familydefault}{\sfdefault}
\usepackage[scaled=0.95]{helvet}

%%%%%% length of the answerline
\setlength\answerlinelength{5.5cm}

%%%%%% change the check-box symbols

\checkboxchar{$\Box$}
\checkedchar{$\blacksquare$}

%%%%%% get rid of extra indentation after the start of a question

\renewcommand{\questionshook}{%
    \setlength{\leftmargin}{0pt}%
    \setlength{\labelwidth}{-\labelsep}%
}

%%%%%% make math also sans-serif
\usepackage{sansmathfonts}
%\renewcommand{\rmdefault}{\sfdefault}

\usepackage{xparse}
\NewDocumentCommand{\solutionfig}{ O{1} O{0.8} m }{
    \begin{solution}[\stretch{#1}]
        $~$\newline \includegraphics[width=#2\hsize]{#3}
    \end{solution}
}

% point symbol

\usepackage{tikz}
\makeatletter
\newcommand\halfpointmark{%
    \textcolor{red}{
        \tikz[baseline=(math.base)] \node[draw,circle,inner sep=1pt] (math) {$\m@th \frac{1}{2}$};%
    }
}
\newcommand\pointmark[1][1]{%
    \textcolor{red}{
        \tikz[baseline=(math.base)] \node[draw,circle,inner sep=1pt] (math) {$\m@th #1$};%
    }
}
\makeatother


%\usepackage{sansmath} % Enables turning on sans-serif math mode, and using other environments
%\sansmath % Enable sans-serif math for rest of document
\sisetup{detect-all} % this also makes siunitx commands use the sans fonts

% change language

%\vqword{Aufgabe}
%\vpword{Punkte}
%\vsword{Ergebnis}
%\vtword{\textbf{Summe}}

%\pointpoints{Punkt}{Punkte}
\pointpoints{Pt.}{Pt.}

% define macros


\def\d{\mathrm{d}}
\def\degree{^\circ}
\def\Rsun{R_\odot}
\def\Msun{M_\odot}
\def\Lsun{L_\odot}
\def\Mjup{M_\mathrm{Jup}}
\def\ajup{a_\mathrm{Jup}}
\def\aearth{a_\mathrm{Earth}}
\def\Rearth{R_\mathrm{Earth}}
\def\Mearth{M_\mathrm{Earth}}
\def\Teff{T_\mathrm{\kern-0.1em ef\kern-0.05em f}}
\def\Mdot{\dot{M}}
\def\me{m_\mathrm{e}}
\def\ne{n_\mathrm{e}}
\def\mp{m_\mathrm{p}}
\def\mn{m_\mathrm{n}}
\def\mH{m_\mathrm{H}}

\def\ion#1#2{#1\,\textsc{#2}}
\def\HI{\ion{H}{i}}
\def\HII{\ion{H}{ii}}
\def\HeI{\ion{He}{i}}
\def\HeII{\ion{He}{ii}}

\def\Mnuc{M_\mathrm{nuc}}
\def\Lnuc{L_\mathrm{nuc}}
\def\Mbolsun{M_\mathrm{bol}^\odot}
\def\Mbolnuc{M_\mathrm{bol}^\mathrm{nuc}}
\def\MdotBH{\dot M_\mathrm{BH}}

\def\vcirc{v_\mathrm{circ}}
\def\tKH{t_\mathrm{KH}}

\def\Egrav{E_\mathrm{grav}}
\def\Edeg{E_\mathrm{degen}}
\def\Ne{N_\mathrm{e}}

\def\tauc{\tau_\mathrm{c}}

\def\Dd{D_\mathrm{d}}
\def\Ds{D_\mathrm{s}}
\def\Dds{D_\mathrm{ds}}

\begin{document}

% ============== PRE-QUESTION TEXT ================

\vspace{0.3in}
\begin{flushright}
  Last name, first name:\enspace\underline{\hspace{7cm}}\\[1em]
  Matrikel-Nr.:         \enspace\underline{\hspace{7cm}}
\end{flushright}
\vspace{0.3em}
\begin{center}
  \begin{infobox}[frametitle={Important Notes:}]
    Answer the questions in the spaces provided on the
    question sheets.  If you run out of room for an answer,
    continue on the back of the page in the empty box and
    note to which question your calculation belongs. At the end of the exam
    (\cpageref{sec:formulae}), there is a collection of formulae that you may
    look up.\\

    The points for each question(-part) are marked on the right margin.\\

    This exam has \numquestions\ questions, for a total of \numpoints\
    points. At least 50\% of the points are needed to pass.

  \end{infobox}
\end{center}
\vspace{0.3em}

% ============== Give some constants ================

\begin{center}
  \begin{Large}
    \bf Quantities\\[-1em]
    \label{sec:quantities}
  \end{Large}
\end{center}
\begin{multicols}{2}
 \begin{align}
  % \rho_\odot                &= \SI{1.41}{g/cm^3} \\
  % c_\mathrm{s}              &= \SI{8.47e4}{cm.s^{-1}} \\
  k_\mathrm{B}       & = \SI{1.38e-16}{erg.K^{-1}}        \\
  \sigma_\mathrm{SB} & = \SI{5.67e-5}{g.K^{-4}.s^{-3}}    \\
  \mathrm{G}         & = \SI{6.67e-8}{cm^3.g^{-1}.s^{-2}} \\
  m_\mathrm{p}       & = \SI{1.67e-24}{g}                 \\
  \mu                & = 2.3                              \\
  \alpha             & = 10^{-2}                          \\
  \rho_\mathrm{s}    & = \SI{2}{g.cm^{-3}}                
 \end{align}
 
 \begin{align}
  V_\mathrm{k}(\SI{1}{au}) & = \SI{2.98e6}{cm.s^{-1}} \\
  \si{au}                  & = \SI{1.5e13}{cm}        \\
  \si{erg}                 & = \si{g.cm^2.s^{-2}}     \\
  M_\star                  & = M_\odot                \\
  M_\odot                  & = \SI{1.99e33}{g}        \\
  M_\mathrm{Jup}           & = \SI{1.9e30}{g}         
 \end{align}
\end{multicols}


\examnewpage

% ============== START OF QUESTIONS ================

\begin{center}
  \begin{Large}
    \bf Questions\\[1em]
    \label{sec:small}
  \end{Large}
\end{center}


\begin{questions}
  % MULTIPLE CHOICE QUESTIONS:
  % we can have multiple choice questions with single-line answers or one answer per line
  %---------------------
  \question [1] The sine of an angle $\theta$ in a right triangle with sides of length $a = 3$ and $c = 5$ is\\
  \begin{oneparcheckboxes}
    \choice         0.4
    \choice         0.5
    \CorrectChoice  0.6
    \choice         0.9
  \end{oneparcheckboxes}
  %---------------------
  \question [2] The cosine of an angle $\theta$ in a right triangle with sides of length $a = 3$ and $c = 5$ is
  \begin{checkboxes}
    \choice         0.4
    \CorrectChoice  0.6
    \choice         0.8
    \choice         1.0
  \end{checkboxes}
  %---------------------
  % quick answers can also be given as answer line
  \question [1] 
  A right triangle has sides of length $a = 3$ and $b = 4$. Calculate the length of the hypotenuse $c$.
  \answerline
  %---------------------

  % here we have a question with a solution box. Depending on how long the solution might be, we can specify the height of the solution box.
  \question [3]
  Explain what the Riemann hypothesis is and proof it.
  \begin{solutionorbox}[4cm] %%%% SETS HEIGHT
    The Riemann hypothesis is a conjecture that the Riemann zeta function has its zeros only at the negative even integers and complex numbers with real part 1/2. It was proposed by Bernhard Riemann in 1859.

    The Riemann hypothesis has not been proven or disproven. It is one of the seven Millennium Prize Problems selected by the Clay Mathematics Institute, each of which carries a US\$1,000,000 prize for the first correct solution.
  \end{solutionorbox}

  \examnewpage

  % a question can also have multiple parts, each with different points
  % we start with some introduction text
  
  \question

  Pythagoras was a Greek mathematician who lived in the 6th century BC. He is credited with the discovery of the Pythagorean theorem, which relates the lengths of the sides of a right triangle. The theorem states that the square of the length of the hypotenuse $c$ is equal to the sum of the squares of the lengths of the other two sides $a$ and $b$:
  \begin{equation}
    c^2 = a^2 + b^2.
    \label{eq:pythagoras}
  \end{equation}
  
  % now we have the parts of the question
  \begin{parts}
    % first part has 2 points and uses two fill-in fields
    % The second fill-in field is wider if needed.
    \part [2] The length of the hypotenuse of a right triangle with sides of length $a = 3$ and $b = 4$ is \fillin[5] while the area of the triangle is \fillin[6][7cm].
    % next part uses an answer line, but can also leave space for a quick computation
    \part [1] What is the length of the side $a$ of a right triangle with hypotenuse $c = 5$ and side $b = 4$?
    \begin{solutionorbox}[\stretch{1}]
      Using the Pythagorean theorem, we have
      \begin{align*}
        a^2 &= c^2 - b^2,\\
        a &= \sqrt{5^2 - 4^2} = 3.
      \end{align*}
    \end{solutionorbox}
    \answerline[3]

    % the third part has just the question and a solution box, but here we need double the space
    % we could also put in something like 2 cm to specify the height of the solution box.
    \part [3] Derive the Pythagorean theorem.

    \begin{solutionorbox}[\stretch{2}]
        Let $a$ and $b$ be the lengths of the two sides of a right triangle, and $c$ the length of the hypotenuse. The area of the square with side length $a + b$ is equal to the sum of the areas of the squares with side lengths $a$ and $b$:
        \begin{align*}
          (a + b)^2 &= a^2 + b^2 + 2ab.
        \end{align*}
        The area of the square with side length $c$ is equal to the sum of the areas of the squares with side lengths $a$ and $b$:
        \begin{align*}
          c^2 &= a^2 + b^2.
        \end{align*}
        Subtracting the first equation from the second gives the Pythagorean theorem:
        \begin{align*}
          c^2 - (a^2 + b^2) &= 0,\\
          c^2 &= a^2 + b^2.
        \end{align*}

    \end{solutionorbox}
  \end{parts}

  %---------------------


  % print a table with the points

  \clearpage

  \begin{center}
    \begin{Large}
      \bf Grading Table
    \end{Large}
    \vspace*{2em}

    \gradetable[v][questions]
    %\gradetable[h][questions]
    %\gradetable[v][pages]
    %\gradetable[h][pages]
  \end{center}
\end{questions}


\begin{center}
  \begin{Large}
    \bf Formulae\\[-1em]
    \label{sec:formulae}
  \end{Large}
\end{center}

% include the formulae
\begin{multicols}{2}
  
% Pythagorean theorem
\begin{equation}
 a^2 + b^2 = c^2
\end{equation}

% Quadratic formula
\begin{equation}
 x = \frac{-b \pm \sqrt{b^2 - 4ac}}{2a}
\end{equation}

% hypergeometric series
\begin{equation}
  \sum_{n=0}^{\infty} \frac{1}{n!} = e
\end{equation}

% Euler's formula
\begin{equation}
  e^{i\theta} = \cos(\theta) + i\sin(\theta)
\end{equation}

% Euler's identity
\begin{equation}
  e^{i\pi} + 1 = 0
\end{equation}

% Fundamental theorem of calculus
\begin{equation}
  \int_a^b f'(x) \, dx = f(b) - f(a)
\end{equation}

\end{multicols}


\end{document}
